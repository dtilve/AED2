\section{Módulo Diccionario Lineal con Hashing (nat,conj(nat))}

\begin{Interfaz}
  \textbf{parámetros formales}\hangindent=2\parindent\\
  \parbox{1.7cm}{\textbf{géneros}}$\alpha$\\
  \parbox[t]{1.7cm}{\textbf{función}}\parbox[t]{.5\textwidth-\parindent-1.7cm}{%
    \InterfazFuncion{Copiar}{\In{k}{$\kappa$}}{$\kappa$}
    {$res \igobs k$}
    [$\Theta(copy(k))$]
    [función de copia de nombres]
  }\\
   	
  \textbf{se explica con}: \tadNombre{Vector($\alpha$), Conjunto Lineal}.

  \textbf{géneros}: \TipoVariable{HashAgXSan}

  \tinterfaz{Operaciones básicas}

%% \In{}{}
%%  OBSERVADORES BASICOS
%% 

  
  \InterfazFuncion{Vacia}{\Inout{v}{vector()}}{}
  [true]
  {$res \igobs <>$}
  [$\Theta(1)$]
  [genera un vector vacio]
  
  \InterfazFuncion{EsVacio?}{\In{v}{vector(nat,conj(nat))}}{bool}
  [true]
  {$res \igobs $}
  [$\Theta(1)$]
  [devuelve true si y solo si v esta vacio]
    
  \InterfazFuncion{Longitud}{\In{v}{vector(nat,conj(nat))}}{nat}
  [true]
  {$res \igobs long(v)$}
  [$\Theta(1)$]
  [devuelve la cantidad de elementos que tiene el vector]
    
  \InterfazFuncion{$\bullet$[$\bullet$]}{\In{v}{vector(nat,conj(nat))}, \In{i}{nat}}{tupla(nat,conj(nat))}
  [i $<$ long(v)]
  {$alias(res \igobs iesimo(v, i))$}
  [$\Theta(1)$]
  [devuelve el elemento que se encuentra en la i-ésima posición de la lista en base 0. Es decir,   $l[i]$   devuelve el elemento que se encuentra en la posición i + 1.]
  
  
%%  GENERADORES
%%	
  
  \InterfazFuncion{Hash}{\Inout{v}{vector(tupla(nat,conj(nat)))}, \In{i}{nat}, \In{tup}{tupla(nat,conj(nat))}}{}
  [$v \igobs v_0 \land i \leq long(v) $]
  {$v \igobs Agregar(v_0,i,tup) $}
  [$\Theta(N_{a})$ la primera vez luego $ \Theta(Log(N_{a})) $]
  [agrega el elemento \textit{tup} a \textit{v}, de forma tal que ocupe la posición \textit{i}]  
 [el elemento a se agrega por copia. Cualquier referencia que se tuviera al vector queda invalidada cuando
long(v) es potencia de 2]
%%  OTRAS OPERACIONES
%%	


\end{Interfaz}

\begin{Representacion}
\tinterfaz{Representación de Diccionario lineal con Hash}

  \begin{Estructura}{HashAgXSanciones}[tHXSanciones]
    \begin{Tupla}[tHXSanciones]%
    
      \tupItem{sanciones}{nat}%
      \tupItem{agentes}{conj(nat)}%
    \end{Tupla} 

  \end{Estructura}
  \tinterfaz{Representación del Conjunto}

	\begin{Estructura}{conjAcotado}[ca]
		\begin{Tupla}[ca]%
			\tupItem{}{}
			\tupItem{}{}
		\end{Tupla}
	\end{Estructura}

  \Rep[ca][c]{($\forall e$: nat)($e$ $\in$ $c$.elementos $\ssi$ $e$ $\geq$ $c$.infimo $\land$ $e$ $<$ $c$.infimo $+$ tam($c$.pertenencia) $\yluego$\\ HaySiguiente?($c$.pertenencia[$e$ $-$ $c$.infimo])) $\yluego$\\ 
  ($\forall e$: nat)($e$ $\in$ $c$.elementos $\impluego$ Siguiente($c$.pertenencia[$e$ $-$ $c$.infimo]) $=$ $e$)}\mbox{}

  ~

  \Abs[ca]{conjAcotado}[e]{c}{Infimo($c$) $=$ $e$.infimo $\land$ Supremo($c$) $=$ $e$.infimo $+$ tam($e$.pertenencia) $-$ $1$ $\land$\\ ConjSuby($c$) $=$ $e$.elementos}


  \tinterfaz{Representación del iterador}

  El iterador del conjunto acotado es simplemente un iterador del conjunto \textit{elementos}, ya que con éste recorremos todos los elementos, más un puntero a la estructura del conjunto, para poder borrar al eliminar el iterador.  

  \begin{Estructura}{itConjAcotado}[itCA]
    \begin{Tupla}[itCA]%
      \tupItem{iter}{itConj(nat)}%
      \tupItem{conj}{puntero(ca)}%
    \end{Tupla}
  \end{Estructura}

  \Rep[itCA][it]{Rep($\ast it$.conj) $\land$ EsPermutacion(SecuSuby($it$.iter), $it$.conj\DRef elementos) }

  ~

  \AbsFc[itCA]{itBi(nat)}[it]{$it$.elementos}


\end{Representacion}

\begin{Algoritmos}

\begin{algorithm}[H]{\textbf{iVacia}(\In{n}{nat})$\to$ $\res$: vector(tupla(nat,info))}
	\begin{algorithmic}
            \State $res \gets (NULL,n)$		\Comment $\Theta(1)$
            
	\end{algorithmic}
\end{algorithm}

\begin{algorithm}[H]{\textbf{iEsVacio}(\In{v}{vector(tupla(nat,info))})$\to$ $\res$: bool}
	\begin{algorithmic}
            \State $res\gets(v.elementos = NULL)$	\Comment $\Theta(1)$
            
	\end{algorithmic}
\end{algorithm}


\begin{algorithm}[H]{\textbf{iLongitud}(\In{v}{vector(tupla(nat,info))})$\to$ $\res$: nat}
	\begin{algorithmic}
            
            \State $res \gets v.longitud$													\Comment $\Theta(1)$
           
	\end{algorithmic}
\end{algorithm}


  \begin{algorithm}[H]{\textbf{i$\bullet$[$\bullet$]}(\In{v}{vector(tupla(nat,info))}, \In{i}{nat})$\to$ $\res$: tupla(nat,info)}
	\begin{algorithmic}
    \State $res \gets v.elementos[i]$       \Comment $\Theta(1)$    
	\end{algorithmic}
\end{algorithm}


\begin{algorithm}[H]{\textbf{iHash}(\Inout{v}{vector(nat,info)}, \In{i}{nat},     \In{tup}{tupla(nat,info)})}
	\begin{algorithmic}
            
            \State $res \gets $													\Comment $\Theta(N_{a})$
           
	\end{algorithmic}
\end{algorithm}
\end{Algoritmos}
