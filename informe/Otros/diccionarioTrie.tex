\section{Módulo diccionario en Trie (string, posición)}

El módulo diccionario en Trie incluye un iterador que permite recorrerlo como un vector donde cada elemento tiene un puntero a cada letra posible (pero sólo uno activado) y un puntero a significado. \par 
No se permiten repeticiones de significados.\par
Siguiendo la nominación del TAD, para este módulo llamamos posición a las tuplas $\langle$x:nat,y:nat$\rangle$. También aprovecharemos la consideración del acceso a strings en su forma vectorial y la función ord, para la correspondencia Char/Nat.

\begin{Interfaz}

% 	\textbf{parámetros formales}\hangindent=2\parindent\\
% 	\parbox{1.7cm}{\textbf{géneros}}\TipoVariable{diccTrie(string, posición)}, \TipoVariable{itDiccTrie(string, posición), \tadNombre{ItBi($\alpha$)}\\
	%\parbox[t]{1.7cm}
%     {\textbf{función}}\parbox[t]{.65 \textwidth-\parindent-1.7cm}{%
%   		\InterfazFuncion{$\bullet = \bullet$}{\In{n_1}{string}, \In{n_2}{string}}{bool}
%     	{$res \igobs (n_1 = n_2)$}
%     	[$\Theta(|n_{min}|)$, donde $|n_{min}|$ es el string más corto (puesto que si hay más caracteres ya no son iguales).]
%     	[función de igualdad de strings]
% 	}\\[2ex]
%   	\parbox[t]{1.7cm}{\textbf{función}}\parbox[t]{.65\textwidth-\parindent-1.7cm}{%
%     	\InterfazFuncion{Copiar}{\In{n}{$string$}}{$string$}
%     	{$res \igobs n$}
%     	[$\Theta(|n|)$, donde $|n|$ es la longitud del string.]
%     	[función de copia de nombres]
%   	}\\[2ex]
%   	\parbox[t]{1.7cm}{\textbf{función}}\parbox[t]{.65\textwidth-\parindent-1.7cm}{%
%     	\InterfazFuncion{Copiar}{\In{p}{$posicion$}}{$posicion$}
%     	{$res \igobs p$}
%     	[$\Theta(1)$]
%     	[función de copia de posiciones]
%   	}

%\textbf{se explica con}: \tadNombre{diccionario$(\kappa, \sigma)$}, \tadNombre{IteradordiccionarioTrie($\alpha$), \tadNombre{Iterador Bidireccional($\alpha$)}.\section{Módulo diccionario en Trie (string, posición)}

  \tinterfaz{Operaciones básicas del diccionario}

  \InterfazFuncion{Vacío}{}{diccTrie$(string,posicion)$}%
  {$res$ $\igobs$ vacío}%
  [$257\Theta(1) = \Theta(1)$]
  [genera un diccionario vacío y lo pasa por referencia.]

  \InterfazFuncion{Definir}{\Inout{d}{diccTrie(string,posicion)}, \In{n}{string}, \In{p}{posicion}}{}
  [$d \igobs d_0 \land \lnot$(def?($d$,$n$))]
  {$d$ $\igobs$ definir($d, n, p$)}
  [$\displaystyle\Theta\left(|n_{m}|\right)$, donde $|n_{m}|$ es la longitud del nombre más largo.]
  [define la clave $n$ con el significado $p$ en el diccionario.]
  []

  \InterfazFuncion{Obtener}{\In{d}{diccTrie($string, posicion$)}, \In{n}{$string$}}{$posicion$}
  [def?($d$, $n$)]
  {alias($res$ $\igobs$ obtener($d$, $n$))}
  [$\Theta(|n_{m}|)$, donde $|n_{m}|$ es la longitud del nombre más largo.]
  [devuelve el significado de la clave $n$ en $d$.]
  [$res$ es modificable puesto que $d$ es modificable.]

  \InterfazFuncion{Definido?}{\In{d}{diccTrie($string, posicion$)}, \In{n}{$string$}}{bool}
  {$res$ $\igobs$ def?($d$, $n$)}
  [$\Theta(CuantasClaves(d)|n_{m}|)$, donde $|n_{m}|$ es la longitud del nombre más largo.]
  [devuelve \texttt{true} si y sólo $n$ está definido en el diccionario.]

  \InterfazFuncion{Borrar}{\Inout{d}{diccTrie($string,posicion$)}, \In{n}{$string$}}{}
  [$d = d_0$ $\land$ def?($d$, $n$)]
  {$d$ $\igobs$ borrar($d_0, n$)}
  [$\Theta(|n_{m}|)$, donde $|n_{m}|$ es la longitud del nombre más largo]
  [elimina la clave $n$ junto con su significado de $d$.]
  [el iterador se indefine si se borran claves sin utilizar la función del iterador Eliminar.]

  \InterfazFuncion{Claves}{\In{d}{diccTrie$(string,posicion)$}}{lista(string)}
  [true]
  {$res \igobs claves(d)$}
  [$\Theta(1)$]
  [devuelve las claves del diccionario por referencia.]

  \tinterfaz{Operaciones básicas del iterador}
  
  \InterfazFuncion{CrearIt}{\In{d}{diccTrie$(string,posicion)$}}{itDiccTrie(puntero))}
  {$res$ $\igobs$ crearItBi(\secuencia{}, $d$)}
  [$\Theta(1)$]
  [crea un iterador bidireccional de la lista que conforma el nodo del diccionario, de forma tal que al pedir \NombreFuncion{Siguiente} se obtenga el primer nodo de $d$.]
  [el iterador se invalida si y sólo si se elimina el elemento siguiente del iterador sin utilizar la funcin \NombreFuncion{EliminarSiguiente}.]
  
    \InterfazFuncion{HaySiguiente}{\In{it}{itDiccTrie(puntero)}}{bool}
  {$res$ $\igobs$ haySiguiente?($it$)}
  [$\Theta(1)$]
  [devuelve \texttt{true} si y sólo si en el iterador todavía quedan elementos para avanzar.]

  \InterfazFuncion{HayAnterior}{\In{it}{itDiccTrie(puntero)}}{bool}
  {$res$ $\igobs$ hayAnterior?($it$)}
  [$\Theta(1)$]
  [devuelve \texttt{true} si y sólo si en el iterador todavía quedan elementos para retroceder.]

  \InterfazFuncion{Siguiente}{\In{it}{itDiccTrie(puntero)}}{puntero}
  [HaySiguiente?($it$)]
  {alias($res$ $\igobs$ Siguiente($it$))}
  [$\Theta(1)$]
  [devuelve el elemento siguiente a la posición del iterador.]
  [$res$ es modificable si y sólo si $it$ es modificable.]

  \InterfazFuncion{Anterior}{\In{it}{itDiccTrie(puntero)}}{puntero}
  [HayAnterior?($it$)]
  {alias($res$ $\igobs$ Anterior($it$))}
  [$\Theta(1)$]
  [devuelve el elemento anterior del iterador.]
  [$res$ es modificable si y slo si $it$ es modificable.]

  \InterfazFuncion{Avanzar}{\Inout{it}{itDiccTrie(puntero)}}{}
  [$it = it_0$ $\land$ HaySiguiente?($it$)]
  {$it$ $\igobs$ Avanzar($it_0$)}
  [$\Theta(1)$]
  [avanza el iterador a la posición siguiente.]

  \InterfazFuncion{Retroceder}{\Inout{it}{itDiccTrie(puntero)}}{}
  [$it = it_0$ $\land$ HayAnterior?($it$)]
  {$it$ $\igobs$ Retroceder($it_0$)}
  [$\Theta(1)$]
  [retrocede el iterador a la posición anterior.]

  \InterfazFuncion{EliminarSiguiente}{\Inout{it}{itDiccTrie(puntero)}}{}
  [$it = it_0$ $\land$ HaySiguiente?($it$)]
  {$it$ $\igobs$ EliminarSiguiente($it_0$)}
  [$\Theta(1)$]
  [elimina de la lista iterada el valor que se encuentra en la posición siguiente del iterador.]

  \InterfazFuncion{EliminarAnterior}{\Inout{it}{itDiccTrie(puntero)}}{}
  [$it = it_0$ $\land$ HayAnterior?($it$)]
  {$it$ $\igobs$ EliminarAnterior($it_0$)}
  [$\Theta(1)$]
  [elimina de la lista iterada el valor que se encuentra en la posición anterior del iterador.]

  \InterfazFuncion{AgregarComoSiguiente}{\Inout{it}{itDiccTrie(puntero)}, \In{a}{puntero}}{}
  [$it = it_0$]
  {$it$ $\igobs$ AgregarComoSiguiente($it_0$, $a$)}
  [$\Theta(1)$]
  [agrega el elemento $a$ a la lista iterada, entre las posiciones anterior y siguiente del iterador, dejando al iterador posicionado de forma tal que al llamar a \NombreFuncion{Siguiente} se obtenga $a$.]
  [el elemento $a$ se agrega por referencia.]

  \InterfazFuncion{AgregarComoAnterior}{\Inout{it}{itDiccTrie(puntero)}, \In{a}{puntero}}{}
  [$it = it_0$]
  {$it$ $\igobs$ AgregarComoAnterior($it_0$, $a$)}
  [$\Theta(1))$]
  [agrega el elemento $a$ a la lista iterada, entre las posiciones anterior y siguiente del iterador, dejando al iterador posicionado de forma tal que al llamar a \NombreFuncion{Anterior} se obtenga $a$.]
  [el elemento $a$ se agrega por referencia.]

\end{Interfaz}

\begin{Representacion}

  \tinterfaz{Representación del diccionario}

La representación que optamos consiste en representar al diccionario como un Trie cuyos nodos son arreglos dimensionables de punteros que constan de 257 posiciones (256 para las letras de las claves y 1 para el significado). Nótese que esto lo hace ineficiente en cuanto espacio, pero eficiente en cuanto al tiempo.\par 
Representamos entonces el conjunto de claves que conforman los string en una lista, que puede recorrerse utilizando iteradores bidireccionales (ver: listas del apunto de módulos básicos).

  \begin{Estructura}{diccTrie}[dt]
    \begin{Tupla}[dt]%
		\tupItem{raiz}{lista(puntero($hijo_{i}$))}
   		\tupItem[\\]{$hijo_{i}$}{lista(puntero($hijo_{k}$))} donde 0 $<=$ i,k $<=$ 256
        \tupItem[\\]{claves}{lista(string)}
    \end{Tupla}
  \end{Estructura}



%  \Rep[][]{}\mbox{}

  %~

 % \Abs[]{}[]{}{}

  \tinterfaz{Representación del iterador}
  
  El iterador es un puntero al puntero siguiente. Este puntero apunta al significado (en este caso una tupla de nat) en el caso en que se llegue al final.
   
  \begin{Estructura}{itDiccTrie}[itdt]
    \begin{Tupla}[itdt]%
		\tupItem{puntero}{puntero(puntero)},
   		\tupItem[\\]{diccionario}{lista(puntero)}
    \end{Tupla}
  \end{Estructura}

\end{Representacion}

\clearpage

\begin{Algoritmos}

\begin{algorithm}[H]{\textbf{iVacío} ()}{$\to$ $\res$: diccTrie}
	\begin{algorithmic}
	\State lst $\gets$ Vacía() \Comment $\Theta(1)$
    \State d.claves $\gets$ Vacía() \Comment $\Theta(1)$
    \State d.significados $\gets$ Vacía() \Comment $\Theta(1)$
    \State i $\gets$ 0
    \While{(i $<$ 257)} \Comment $\Theta(1)$
    \State AgregarAtras(lst,NULL) \Comment $\Theta(1)$
    \State i = i + 1 	\Comment $\Theta(1)$
    \EndWhile
    \State res.raíz $\gets$ lst 	\Comment $\Theta(1)$
    \State res $\gets$ $\&$raiz
    \medskip
	\Statex \underline{Complejidad:} $\Theta(1)$
    \Statex \underline{Justificación:} La creación de una lista de dimensión fija es constante, el agregado a la misma del NULL como elemento es $\Theta(1)$ pues es un macro para un tipo básico (bien se podría agregar un 0) . El resultado se devuelve por referencia, no por copia.
	\end{algorithmic}
\end{algorithm}

\begin{algorithm}[H]{\textbf{iDefinir}(\Inout{d}{diccTrie}, \In{n}{string}, \In{p}{posición})}
	\begin{algorithmic}
    \State current $\gets$ d.raíz 			\Comment $\Theta(1)$
    \State i = 0							\Comment $\Theta(1)$
    \While{($i < Longitud(n)$)}				\Comment $\Theta(1)$
    	\If{(current[ord(n[i])] == NULL)}	\Comment $\Theta(1)$
            \State new $\gets$ NuevoNodo()	\Comment $\Theta(1)$
            \State current.$hijo_{i}$ = new
            \State current[ord(n[i])] $\gets$ $\&$new \Comment $\Theta(1)$
            \State current = new			\Comment $\Theta(1)$
        \Else
        	\State current $\gets$ current[ord(n[i])] \Comment $\Theta(1)$
		\EndIf
    \State i = i + 1						\Comment $\Theta(1)$
    \EndWhile
    \State current[256] $\gets$ p	\Comment $\Theta(1)$
    \State d.Claves$\to$AgregarAtrás(n)		\Comment $\Theta(1)$
    \medskip
	\Statex \underline{Complejidad:} $\Theta(|n_{m}|)$
    \Statex \underline{Justificación:} Primero una observación particular: como n es un vector(char), su longitud puede conseguirse $\Theta(1)$. Luego, se tiene que en el peor caso se deben definir cada uno de los nuevos chars, eso hace que el loop que se tiene en $\Theta(1)$ se repita la cantidad de veces que es el largo del string.
	\end{algorithmic}
\end{algorithm}

\begin{algorithm}[H]{\textbf{iObtener(\In{d}{diccTrie}, \In{n}{string})}}{$\to$ $\res$:{posición}}
	\begin{algorithmic}
 	\State current $\gets$ d.raíz 			\Comment $\Theta(1)$
    \State i = 0							\Comment $\Theta(1)$
    \While{($i < Longitud(n)$)}				\Comment $\Theta(1)$
    	\State current $\gets$ current[ord(n[i])] \Comment $\Theta(1)$
        \State i = i + 1						\Comment $\Theta(1)$
    \EndWhile
    \State res $\gets$ current[256] \Comment $\Theta(1)$
  	\Statex \underline{Complejidad:} $\Theta(|n_{m}|)$
    \Statex \underline{Justificación:} Obtener el significado implica buscar un nuevo char en $\Theta(1)$ por cada uno de los chars que compone la clave.
	\end{algorithmic}
\end{algorithm}

\begin{algorithm}[H]{\textbf{iDefinido?(\In{d}{diccTrie}, \In{n}{string})}}{$\to$ $\res$:{bool}}
	\begin{algorithmic}
	\State current $\gets$ d.raíz 			\Comment $\Theta(1)$
    \State i = 0							\Comment $\Theta(1)$
    \While{($i < Longitud(n)$ $\&\&$ current $!=$ NULL )}				\Comment $\Theta(1)$
    	\State current $\gets$ current[ord(n[i])] \Comment $\Theta(1)$
        \State i = i + 1						\Comment $\Theta(1)$
    \EndWhile
    \State res $\gets$ (i == Longitud(n))
    \Statex \underline{Complejidad:} $\Theta(|n_{m}|)$
    \Statex \underline{Justificación:} Comprobar si está definido implica buscar los chars que componen la clave en $\Theta(1)$ por cada uno en el peor caso (es decir, en el que la búsqueda del char siempre haya sido generadora de un nuevo ciclo).
	\end{algorithmic}
\end{algorithm}

\begin{algorithm}[H]{\textbf{iBorrar(\Inout{d}{diccTrie}, \In{n}{string})}}
	\begin{algorithmic} 
    \State itd $\gets$ CrearIt(d)
    \State Siguiente(itd)
    \State i = 0
    \While{($i < Longitud(n)$) $\&\&$ haySiguiente(itD)}
    \State current $\gets$ current[ord(n[i])]
    \State itD $\gets$ Siguiente(itD)
    \State i = i + 1 
    \EndWhile
    \State j = 0
    \While{($j < Longitud(n)-1$)}
    \State current $\gets$ current[ord(n[i])]
    \State EliminarSiguiente(itD)
    \State Anterior(itD)
    \State j = j + 1 
    \EndWhile 
    \Statex \underline{Complejidad:} $\Theta(|n_{m}|)$
    \Statex \underline{Justificación:} Es necesario llegar hasta el final de la clave e ir eliminando los elementos de a uno, por lo que se recorre con el iterador (para evitar indefiniciones) en $\Theta(1)$, como mucho, el nombre más largo, es decir, $n_{m}$ veces $\Theta(1)$.
	\end{algorithmic}
\end{algorithm}

% \begin{algorithm}[H]{\textbf{iClaves}(\In{d}{diccTrie}}{$\to$ $\res$:{lista(string)}}
% 	\begin{algorithmic}
%     \State res $\gets$ $\&$d.claves
% 	\end{algorithmic}
% \end{algorithm}

% \begin{algorithm}[H]{\textbf{iCuantasClaves}(\Inout{d}{diccTrie}, \In{n}{string}, \In{p}{posición})}
% 	\begin{algorithmic}
%     \State asdf
% 	\end{algorithmic}
% \end{algorithm}

% \begin{algorithm}[H]{\textbf{$=_{operator}$}(\Inout{d}{diccTrie}, \In{n}{string}, \In{p}{posición})}
% 	\begin{algorithmic}
%     \State asdf
% 	\end{algorithmic}
% \end{algorithm}

\tinterfaz{Auxiliares}

\begin{algorithm}[H]{\textbf{NuevoNodo}()}{$\to$ $\res$: lista(puntero(NULL))}
	\begin{algorithmic}
        \State nodo $\gets$ Vacía() \Comment $\Theta(1)$
        \State i $\gets$ 0
        \While{(i $<$ 257)} \Comment $\Theta(1)$
        \State AgregarAtras(nodo,NULL) \Comment $\Theta(1)$
        \State i = i + 1 	\Comment $\Theta(1)$
        \EndWhile
        \State res $\gets$ nodo \Comment $\Theta(1)$
    	\Statex \underline{Complejidad:} $\Theta(1)$
    	\Statex \underline{Justificación:} Como los tiempos son constantes puesto que hay una limitación en la cantidad de posiciones para la lista y los elementos son básicos, se puede obtener una lista "nodo" nueva en $\Theta(1)$.
	\end{algorithmic}

\end{algorithm}
\end{Algoritmos}