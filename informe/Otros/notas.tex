%Holaaa :D

%queria escribir las estructuras que tenemos asi veo mejor que falta

\tupItem{campus}				{puntero(matriz(char))} %// en esta aridad faltan los 3 punteros??
% \tupItem[\\]{estudiantes}		{conj(estudiantes:string)} 
% \tupItem[\\]{hippies}			{conj(hippies:string)}  
% \tupItem[\\]{agentes}			{conj(agentes:nat)} %/////// para mi este hay que reemplazarlo por arregloAgentes y que el puntero del Campus que va a un agente valla a arregloAgentes
\tupItem[\\]{diccHippies}		{diccTrie(nombre:string, pos:tupla(nat,nat))} %//estos son los trie
\tupItem[\\]{diccEstudiantes}	{diccTrie(nombre:string, pos:tupla(nat,nat))} %/estos son los trie
\tupItem[\\]{diccAgentes}		{diccAg(placa:nat, info)} %//ordenados por placa
% \tupItem[\\]{arregloAgentes}	{arrAg(itDiccAgentes)} %//este es el hash - este iterador apunta a una posicion de diccAgentes
% \tupItem[\\]{listaSanciones}	{listSanc(NodoSancion)} %//ordenada por #sancion - para mi se representa con una lista simplemente enlazada con puntero al primero y el ultimo es suficiente
% \tupItem[\\]{arregloSanciones}	{arrSanc(placa:nat, itListSanc)} %//ordenado por #sancion - el iterador apunta al nodo  de listaSanciones que correponde por igualdad de #sancion
% \tupItem[\\]{másVigilante}		{itDiccAgentes} %// este iterador apunta a una posicion de diccAgentes
% \tupItem[\\]{flagCaptura}		{bool}
% \tupItem[\\]{sanciones}			{diccHashS(placa:nat, sanciones:nat)}

Para simplificar la lectura descartamos $\Pi_i$ y definimos:\par
\begin{Tupla}[info]
  \tupItem{posicion}{tupla(nat,nat)}
  \tupItem{sanciones}{$nat$}
  \tupItem{capturas}{nat}
  \tupItem{iteradorAListaSanciones}{itListSanc} %//este iterador apunta al nodo  de listaSanciones que correponde por igualdad de #sancion
\end{Tupla}

\begin{Tupla}[NodoSancion]
  \tupItem{siguiente}{*NodoSancion}
  \tupItem{numeroSancion}{nat}
  \tupItem{agentes}{conj(agentes:nat)}
\end{Tupla}
    